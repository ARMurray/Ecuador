%% March 2018
%%%%%%%%%%%%%%%%%%%%%%%%%%%%%%%%%%%%%%%%%%%%%%%%%%%%%%%%%%%%%%%%%%%%%%%%%%%%
% AGUJournalTemplate.tex: this template file is for articles formatted with LaTeX
%
% This file includes commands and instructions
% given in the order necessary to produce a final output that will
% satisfy AGU requirements, including customized APA reference formatting.
%
% You may copy this file and give it your
% article name, and enter your text.
%
%
% Step 1: Set the \documentclass
%
% There are two options for article format:
%
% PLEASE USE THE DRAFT OPTION TO SUBMIT YOUR PAPERS.
% The draft option produces double spaced output.
%

%% To submit your paper:
\documentclass[draft,linenumbers]{agujournal2018}
\usepackage{apacite}
\usepackage{url} %this package should fix any errors with URLs in refs.
%%%%%%%
% As of 2018 we recommend use of the TrackChanges package to mark revisions.
% The trackchanges package adds five new LaTeX commands:
%
%  \note[editor]{The note}
%  \annote[editor]{Text to annotate}{The note}
%  \add[editor]{Text to add}
%  \remove[editor]{Text to remove}
%  \change[editor]{Text to remove}{Text to add}
%
% complete documentation is here: http://trackchanges.sourceforge.net/
%%%%%%%


%% Enter journal name below.
%% Choose from this list of Journals:
%
% JGR: Atmospheres
% JGR: Biogeosciences
% JGR: Earth Surface
% JGR: Oceans
% JGR: Planets
% JGR: Solid Earth
% JGR: Space Physics
% Global Biogeochemical Cycles
% Geophysical Research Letters
% Paleoceanography and Paleoclimatology
% Radio Science
% Reviews of Geophysics
% Tectonics
% Space Weather
% Water Resources Research
% Geochemistry, Geophysics, Geosystems
% Journal of Advances in Modeling Earth Systems (JAMES)
% Earth's Future
% Earth and Space Science
% Geohealth
%
% ie, \journalname{Water Resources Research}

\journalname{Geophysical Research Letters}


\usepackage{soulutf8}

\begin{document}

%% ------------------------------------------------------------------------ %%
%  Title
%
% (A title should be specific, informative, and brief. Use
% abbreviations only if they are defined in the abstract. Titles that
% start with general keywords then specific terms are optimized in
% searches)
%
%% ------------------------------------------------------------------------ %%

% Example: \title{This is a test title}

\title{Spatiotemporal Assessments of Greenhouse Gas Concentration and Flux in
Headwater Tropical Streams}

%% ------------------------------------------------------------------------ %%
%
%  AUTHORS AND AFFILIATIONS
%
%% ------------------------------------------------------------------------ %%

% Authors are individuals who have significantly contributed to the
% research and preparation of the article. Group authors are allowed, if
% each author in the group is separately identified in an appendix.)

% List authors by first name or initial followed by last name and
% separated by commas. Use \affil{} to number affiliations, and
% \thanks{} for author notes.
% Additional author notes should be indicated with \thanks{} (for
% example, for current addresses).

% Example: \authors{A. B. Author\affil{1}\thanks{Current address, Antartica}, B. C. Author\affil{2,3}, and D. E.
% Author\affil{3,4}\thanks{Also funded by Monsanto.}}

\authors{
Andrew R. Murray
\affil{1}
Keridwen Whitmore
\affil{1}
Diego Riveros-Iregui
\affil{1}
Andrea Encalada
\affil{2}
Esteban Suarez
\affil{2}
Gonzalo Rivas-Torres
\affil{2}
}


% \affiliation{1}{First Affiliation}
% \affiliation{2}{Second Affiliation}
% \affiliation{3}{Third Affiliation}
% \affiliation{4}{Fourth Affiliation}

\affiliation{1}{University of North Carolina - Chapel Hill Department of Geography}
\affiliation{2}{Universidad de San Francisco de Quito}
%(repeat as many times as is necessary)

%% Corresponding Author:
% Corresponding author mailing address and e-mail address:

% (include name and email addresses of the corresponding author.  More
% than one corresponding author is allowed in this LaTeX file and for
% publication; but only one corresponding author is allowed in our
% editorial system.)

% Example: \correspondingauthor{First and Last Name}{email@address.edu}
\correspondingauthor{Andrew Murray}{armurray@live.unc.edu}

%% Keypoints, final entry on title page.

%  List up to three key points (at least one is required)
%  Key Points summarize the main points and conclusions of the article
%  Each must be 100 characters or less with no special characters or punctuation

% Example:
% \begin{keypoints}
% \item	List up to three key points (at least one is required)
% \item	Key Points summarize the main points and conclusions of the article
% \item	Each must be 100 characters or less with no special characters or punctuation
% \end{keypoints}

\begin{keypoints}
\item Streams in the Páramo region are carbon sources.
\item Variability in carbon evasion is driven by discharge and slope.
\item CO2 evasion can be quantified with a minimal appraoch.
\end{keypoints}

%% ------------------------------------------------------------------------ %%
%
%  ABSTRACT
%
% A good abstract will begin with a short description of the problem
% being addressed, briefly describe the new data or analyses, then
% briefly states the main conclusion(s) and how they are supported and
% uncertainties.
%% ------------------------------------------------------------------------ %%

%% \begin{abstract} starts the second page

\begin{abstract}
Long thought to be a carbon sink, the Páramo region of the Andes may be
a significant carbon source. We evaluated the spatiotemporal dynamics of
CO2 concentration and flux in a high-elevation stream. We measured
15-min dissolved CO2, O2, and discharge, across a wetland-stream
transition, characteristic of tropical, alpine environments.
\end{abstract}
\noindent{\bf Plain language summary}\vskip-\parskip

\noindent{Streams emit carbon dioxide to the atmosphere, but how much? Recent
research suggests that steaper streams emit more carbon dioxide since
they are more turbulent, but the amount of carbon dioxide a stream emits
is also dependent on the carbon content of the local ecosystem. The
paramo, which is a region in the high Andes mountains in Peru, Ecuador
and Columbia, above the treeline (\textasciitilde{}10,000 ft) and below
the snow line (\textasciitilde{}16,000 ft), is made up of extensive
peatlands which hold enormous amounts of carbon. In the past, it was
believed that the paramo took in carbon from the atmosphere and stored
it however, recent research suggests that the region has reversed course
and is now emitting more carbon into the atmosphere than it is taking
in. A principal way that carbon is moved from the soils into the
atmosphere is through the water cycle. We set up a sensor network in a
stream in the Ecuadorian paramo to measure changes in carbon dioxide and
dissolved oxygen, while also recording stream discharge. We then took
this data and created a method for determining how much carbon dioxide
is being emitted from the stream to the atmosphere and how that relates
to discharge, steepness of the stream, and precipitation.}
\vskip18pt
\begin{verbatim}
## -- Attaching packages ----------------------------------------------------------------------------------------------- tidyverse 1.3.0 --
\end{verbatim}

\begin{verbatim}
## v ggplot2 3.2.1     v purrr   0.3.3
## v tibble  2.1.3     v dplyr   0.8.4
## v tidyr   1.0.2     v stringr 1.4.0
## v readr   1.3.1     v forcats 0.5.0
\end{verbatim}

\begin{verbatim}
## -- Conflicts -------------------------------------------------------------------------------------------------- tidyverse_conflicts() --
## x dplyr::filter() masks stats::filter()
## x dplyr::lag()    masks stats::lag()
\end{verbatim}

\begin{verbatim}
## 
## Attaching package: 'plotly'
\end{verbatim}

\begin{verbatim}
## The following object is masked from 'package:ggplot2':
## 
##     last_plot
\end{verbatim}

\begin{verbatim}
## The following object is masked from 'package:stats':
## 
##     filter
\end{verbatim}

\begin{verbatim}
## The following object is masked from 'package:graphics':
## 
##     layout
\end{verbatim}

\begin{verbatim}
## 
## Attaching package: 'scales'
\end{verbatim}

\begin{verbatim}
## The following object is masked from 'package:purrr':
## 
##     discard
\end{verbatim}

\begin{verbatim}
## The following object is masked from 'package:readr':
## 
##     col_factor
\end{verbatim}

\begin{verbatim}
## 
## Attaching package: 'lubridate'
\end{verbatim}

\begin{verbatim}
## The following object is masked from 'package:base':
## 
##     date
\end{verbatim}

\begin{verbatim}
## here() starts at /proj/diegorilab/users/Andrew/Ecuador
\end{verbatim}

\begin{verbatim}
## 
## Attaching package: 'here'
\end{verbatim}

\begin{verbatim}
## The following object is masked from 'package:lubridate':
## 
##     here
\end{verbatim}

\section{Introduction}

It is well known that CO2 flux from freshwater streams impacts the
global carbon cycle, however our ability to quantify these affects at a
variety of spatial and temporal scales remains limited. The Páramo has
historically been thought to be a carbon sink however, recent research
suggests that it is now a source of carbon to the atmosphere
\citep{carrillo2019breathing}. We tested two methods of quantifying CO2
evasion in headwater streams in the Páramo region of Ecuador and discuss
the results and implications of our findings. We utilized eosense EosFD
flux chamber sensors to measure CO2 flux directly from streams using
in-stream flotation devices. Additionally, we used multiple Vaisala CO2
concentration sensors, fitted with waterproof PTFE sleeves to monitor
change in in-stream concentration and calculated flux by converting
concentration to mass per area time using the formula:

\begin{linenomath*}
\begin{equation}
f=\frac{\Delta ppm}{A}*.0018*\frac{10^6}{44.01}*Q+(R-P)
\end{equation}
\end{linenomath*}

Where \(f\) equals flux in \(µmol\:m^{-2}s^{-1}\), \(\Delta ppm\) is the
change in CO2 ppm between sensors, \(R\) is stream respiration in ppm,
\(P\) is gross primary production in ppm, A is the stream surface area
between sensors in \(m^2\), .0018 is the conversion coefficient for
\(CO_2\) from ppm to \(g\:m^{-3}\) and 44.01 is the molar mass of
\(CO_2\).

\subsection{Background}

The Páramo region is characterized by high elevation, mountainous
wetlands that are connected by turbulent streams. Connectivity can be
seasonal and dependent upon precipitation intensity. Our specific site
lies on the western edge of the Andean continental divide. The site
receives mean monthly precipitation of 121 mm (Standard deviation =
61mm). Precipitation is moderately seasonal with maximum precipitation
typically occurring in July and minimum precipitation in February. The
site experiences some level of precipitation on 83.5\% of days. Mean
daily temperature is 50 C. We instrumented a reach approximately 140
meters in length which serves as a direct outlet to a wetland with an
area of 2.3 Ha as well as multiple additional wetlands further upstream.
Four sensor clusters were installed at various intervals representing
various stream gradients (figure 1). A waterfall measuring
\textasciitilde{}4m exists between station 3 and 4.

\section{Methods}

Four Vaisala GM-220 CO2 sensors were used to measure dissolved in-stream
CO2 concentrations. We used waterproof PTFE sleeves, following the
method presented in Johnson., et al.(2010). The Vaisala sensors, which
have a range of 0-10,000 ppm, have a margin of error of 150 ppm (1.5\%).
To minimize error, all four sensors were initially collocated to
establish offset values. Two eosFD (EoSense) chamber sensors were
positioned in the stream using custom made floating platforms. One
sensor was located between Station 1 and 2, and the other was collocated
with station 4. To estimate evasion from the reach, we quantified CO2
change between station 1 and 4. Here is text on Materials and Methods.

\hypertarget{htmlwidget-3c87f60362e56efce783}{}

Please caption every figure

Do not use bulleted lists; enumerated lists are okay. Use \#. for list
for a cleaner LaTeX output.

\begin{enumerate}
\item
  First element
\item
  Second element
\end{enumerate}

\subsection{A descriptive heading about methods}

\section{Data}

Or section title might be a descriptive heading about data

As of 2018 we recommend use of the TrackChanges package to mark
revisions. The trackchanges package adds five new LaTeX commands:

\textbackslash{}note{[}editor{]}\{The note\}

\textbackslash{}annote{[}editor{]}\{Text to annotate\}\{The note\}

\textbackslash{}add{[}editor{]}\{Text to add\}

\textbackslash{}remove{[}editor{]}\{Text to remove\}

\textbackslash{}change{[}editor{]}\{Text to remove\}\{Text to add\}

complete documentation is here: http://trackchanges.sourceforge.net/

\section{Results}

Or section title might be a descriptive heading about the results

Enter Figures and Tables near as possible to where they are first
mentioned: DO NOT USE \textbackslash{}psfrag or
\textbackslash{}subfigure commands. DO NOT USE
\textbackslash{}newcommand, \textbackslash{}renewcommand, or
\textbackslash{}def, etc.

\begin{figure}[h]
\includegraphics{StreamFluxArticle_files/figure-latex/unnamed-chunk-2-1} \caption{Please caption every figure}\label{fig:unnamed-chunk-2}
\end{figure}

Example table

\begin{table}
 \caption{Time of the Transition Between Phase 1 and Phase 2$^{a}$}
 \centering
 \begin{tabular}{l c}
 \hline
  Run  & Time (min)  \\
 \hline
   $l1$  & 260   \\
   $l2$  & 300   \\
   $l3$  & 340   \\
   $h1$  & 270   \\
   $h2$  & 250   \\
   $h3$  & 380   \\
   $r1$  & 370   \\
   $r2$  & 390   \\
 \hline
 \multicolumn{2}{l}{$^{a}$Footnote text here.}
 \end{tabular}
 \end{table}

AGU prefers the use of \{sidewaystable\} over \{landscapetable\} as it
causes fewer problems.

\begin{sidewaysfigure}[h]
\includegraphics{StreamFluxArticle_files/figure-latex/unnamed-chunk-3-1} \caption{Please caption every figure}\label{fig:unnamed-chunk-3}
\end{sidewaysfigure}

\begin{sidewaystable}
\caption{Caption here}
\label{tab:signif_gap_clos}
\begin{tabular}{ccc}
one&two&three\\
four&five&six
\end{tabular}
\end{sidewaystable}

If using numbered lines, please surround equations with
\textbackslash{}begin\{linenomath*\}\ldots{}
\textbackslash{}end\{linenomath*\}

\begin{linenomath*}
\begin{equation}
y|{f} \sim g(m, \sigma)
\end{equation}
\end{linenomath*}

\section{Conclusions}

\appendix
\section{Here is a sample appendix}

Optional Appendix goes here

Optional Glossary, Notation or Acronym section goes here:

Glossary is only allowed in Reviews of Geophysics

\begin{glossary}
\term{Term}
 Term Definition here
\term{Term}
 Term Definition here
\term{Term}
 Term Definition here
\end{glossary}

\begin{acronyms}
\acro{Acronym}
 Definition here
\acro{EMOS}
 Ensemble model output statistics
\acro{ECMWF}
 Centre for Medium-Range Weather Forecasts
\end{acronyms}

\begin{notation}
\notation{$a+b$} Notation Definition here
\notation{$e=mc^2$}
Equation in German-born physicist Albert Einstein's theory of special
relativity that showed that the increased relativistic mass ($m$) of a
body comes from the energy of motion of the body—that is, its kinetic
energy ($E$)—divided by the speed of light squared ($c^2$).
\end{notation}

\acknowledgments

Data, as well as all scripts for conducting this analysis are hosted on
GitHub and can be obtained at:
https://github.com/ARMurray/Ecuador/tree/master/Articles/StreamFluxArticle

This work was funded by the National Science Foundation under award
1847331: \emph{The role of small wetland connectivity in controlling
greenhouse gas emissions and downstream carbon fluxes from headwater
tropical streams.}

The authors would like to thank:

La Universidad de San Francisco de Quito for their material and
institutional support throughout this research.

\section{Reminders}

\subsection{How to cite}

Please use ONLY \textbackslash{}citet and \textbackslash{}citep for
reference citations. DO NOT use other cite commands (e.g.,
\textbackslash{}cite, \textbackslash{}citeyear, \textbackslash{}nocite,
\textbackslash{}citealp, etc.). Example \textbackslash{}citet and
\textbackslash{}citep: \ldots{}as shown by \citet{Levitus2012},
\citet{Nuncio2011} and \citet{Raphael2004} \ldots{}as shown by
\citep{Levitus2012}, \citep{Nuncio2011}, \citep{Raphael2004}.
\ldots{}has been shown
\citep[e.g.,][]{Levitus2012, Nuncio2011, Raphael2004}.

\bibliography{references.bib}


\end{document}
